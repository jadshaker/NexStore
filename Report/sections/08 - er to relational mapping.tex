\section{ER to Relational Mapping}

\subsection{Mapping of Strong Entity Types}

\begin{itemize}
  \item For each regular (strong) entity type E in the ER schema, create a relation R that includes all the simple (atomic) attributes of E.
  \item Choose one of the key attributes of E as the primary key for R.
  \item If the chosen key of E is composite, the set of simple attributes that form it will together form the primary key of the relation R. \cite{slides}
\end{itemize}

\subsubsection{Branch}

Branch (\underline{Phone\_number}, Name, Country, State, City, Street, Building, Apartment)

We created the relation Branch including various attributes of the strong entity Branch. We included all the simple (atomic) attributes of Branch that are Name and Phone\_number. Moreover, we decomposed the composite attribute Address into simple attributes as country, state, city, street, building , and apartment. Furthermore, we didn't include the composite multi-valued attribute which is Work\_hours as we'll create its own relation later. Finally, we chose the key attribute Phone\_number to be the primary key that uniquely identifies the relation Branch.

\subsubsection{Category}

Category (\underline{Name}, Description)

We created the relation Category including the two atomic attributes of the strong entity Category that are Name and Description. Additionally, we chose the key attribute Name as the primary key.

\subsubsection{Coupon}

Coupon (\underline{Code}, Description, Discount\_percent, Times\_used, Minimum\_order\_amount,\\
Maximum\_order\_amount, Usage\_limit, Valid\_from, Valid\_to)

We created the relation Coupon including attributes of the strong entity Coupon. We included all atomic attributes which are Code, Description, Discount\_percent, Times\_used, Minimum\_order\_amount, Maximum\_order\_amount, and Usage\_limit. Furthermore, we decomposed the composite attribute Valid into two simple attributes which are Valid\_from and Valid\_to. Finally, we chose the key attribute Code as a primary key.

\subsubsection{Customer}

Customer (\underline{Phone\_number}, Email, First\_name, Last\_name, Gender, Registration\_date, Password\_hashed, Date\_of\_birth, Country, State, City, Street, Building, Apartment)

We created the relation Customer including different attributes of the strong entity Customer. We included all the atomic attribute which are Phone\_number, Email, Gender, Registration\_date, Password\_hashed and Date\_of\_birth. Moreover, we decomposed the two composite attribute Name and Address as First\_name and Last\_name and Country, State, City, Street, Building and Apartment, respectively.

\subsubsection{Department}

Department (\underline{Name}, Number\_of\_employees)

We created the relation Department including various attributes of the strong entity Department. We included all the simple attributes which are Name and Number\_of\_employees, such that Name acts a primary key and Number\_of\_employees as a derived attribute. Furthermore, we didn't include the multi-valued attribute which is Locations as we'll create its own relation later.


\subsubsection{Driver}

Driver (\underline{License\_number}, Driving\_experience\_years, License\_expiry\_date)

We created the relation Driver including all the atomic attributes of the strong entity Driver which are: the key attribute which is License\_number as a primary key, Driving\_experience\_years and License\_expiry\_date.

\subsubsection{Employee}

Employee (\underline{SSN}, Position, Salary, Hire\_date, Gender, Date\_of\_birth, Email, First\_name, Last\_name, \\
Phone\_number, Country, State, City, Street, Building, Apartment)

We created the relation Employee including various attributes of the regular entity Employee. We included all the simple attributes which are: SSN, Position, Salary, Hire\_date, Gender, Date\_of\_birth, Email and Phone\_number. Moreover, we decomposed the two composite attribute Name and Address as First\_name and Last\_name and Country, State, City, Street, Building and Apartment, respectively. Finally, we chose the key attribute SSN as the primary key.

\subsubsection{Order}

Order (\underline{Order\_id}, Notes, Payment\_method, Total\_amount, Is\_online)

We created the relation Order including all the atomic attributes of the strong entity Order which are: the key attribute which is Order\_id as a primary key, Notes, Payment\_method, Total\_amount and Is\_online.

\subsubsection{Product}

Product (\underline{SKU}, Name, Price, Description, Weight, Brand, Width, Height, Length)

We created the relation Product including various attributes of the regular entity Product. We included all the simple attributes which are SKU, Name, Price, Description, Weight and Brand. Moreover, we decomposed the composite attribute Dimensions into three atomic attributes which are: Width, Height and Length. Furthermore, we didn't include the multi-valued attributes which are Image\_URLs and Colors as we'll create relations from them later. Finally, we chose the key attribute SKU as the primary key.

\subsubsection{Supplier}

Supplier (\underline{Website}, Supplier\_name, Contact\_person\_email, \\
Contact\_person\_first\_name, Contact\_person\_last\_name, Contact\_person\_phone\_number )

We created the relation Supplier including various attributes of the regular entity Supplier. We included all the simple attributes which are: Supplier\_name and Website. Moreover, we decomposed the composite attribute into two atomic attributes and a composite attribute, which are Email and Phone\_number and Name, respectively. Furthermore, we decomposed Name into two atomic attributes which are First\_name and Last\_name. Finally, we chose the key attribute Website as the primary key.

\subsubsection{Support Ticket}

Support\_Ticket (\underline{Ticket\_id}, Description, Subject, Status, Priority)

We created the relation Support\_Ticket including attributes of the regular entity Support\_Ticket. We included all the simple attributes which are: Ticket\_id, Description, Subject, Status and Priority.  Finally, we chose the key attribute Ticket\_id as the primary key.

\subsection{Mapping of Weak Entity Types}

\begin{itemize}
  \item	According to Elmasri and Navathe, in Fundamentals of Database Systems (2015), "For each weak entity type W in the ER schema with owner entity type E, create a relation R \& include all simple attributes (or simple components of composite attributes) of W as attributes of R.
  \item	Also, include as foreign key attributes of R the primary key attribute(s) of the relation(s) that correspond to the owner entity type(s).
  \item	The primary key of R is the combination of the primary key(s) of the owner(s) and the partial key of the weak entity type W, if any." \cite{elmasri}
\end{itemize}

\subsubsection{Dependent}

Dependent (\underline{\textit{Employee\_SSN}}, \underline{Name}, Gender, Date\_of\_birth, Relationship)

We created a relation Dependent for the weak entity type Dependent with employee entity type Employee. We included all the atomic attributes of Dependent which are Name, Gender, Date\_of\_birth, and Relationship. Moreover, we created the foreign key Employee\_SSN which references to the primary key of Employee which is SSN. Finally, we assigned the tuple Employee\_SSN and the weak attribute Name as the primary key of this relation.

\subsection{Mapping of Binary 1:1 Relationship Types}

\begin{itemize}
  \item According to Elmasri and Navathe, in Fundamentals of Database Systems (2015), "For each binary 1:1 relationship type R in the ER schema, identify the relations S and T that correspond to the entity types participating in R.
  \item Foreign Key approach: Choose one of the relations-say S-and include a foreign key in S the primary key of T. It is better to choose an entity type with total participation in R in the role of S." \cite{elmasri}
\end{itemize}

\subsubsection{Redeem}

Coupon (\underline{Code}, Description, Discount\_percent, Times\_used, Minimum\_order\_amount, \\
Maximum\_order\_amount, Usage\_limit, Valid\_from, Valid\_to, \textit{Order\_ID}, \\
Discount\_amount\_ Redeem\_date)

The 1:1 relationship Redeem is mapped by choosing the participating entity type Coupon to serve in the role of S, because both participating entity types have a partial participation in the Redeem relationship type, so, it doesn't matter which one we choose. Moreover, we chose Order\_ID as the foreign key referencing to the primary key of Order. Finally, we added all the atomic attributes of the relationship Redeem to the relation Coupon.

\subsubsection{Manages Branch}

Branch (\underline{Phone\_number}, Name, Country, State, City, Street, Building, Apartment, \textit{Employee\_SSN})

The 1:1 relationship Manages\_branch is mapped by choosing the participating entity type Branch to serve in the role of S, because its participation in the Manages\_branch relationship type is total. Moreover, we added the foreign key Employee\_SSN referencing to the primary key of Employee.

\subsubsection{Is Driver}

Driver (\underline{License\_number}, Driving\_experience\_years, License\_expiry\_date, \textit{Employee\_SSN})

The 1:1 relationship Is\_driver is mapped by choosing the participating entity type Driver to serve in the role of S, because its participation in the Is\_driver relationship type is total. Moreover, we add the foreign key Employee\_SSN referencing to the primary key of Employee.

\subsubsection{Manages Department}

Department(\underline{Name}, Number\_of\_employees, \textit{Employee\_SSN}, Manager\_start\_date)

The 1:1 relationship Manages\_department is mapped by choosing the participating entity type Department to serve in the role of S, because its participation in the Manages\_department relationship type is total. Moreover, we add the foreign key Employee\_SSN referencing to the primary key of Employee. Finally, we added the atomic attribute of the relationship Manages\_department to the relation Coupon.

\subsection{Mapping of Binary 1:N Relationship Types}

\begin{itemize}
  \item According to Elmasri and Navathe, in Fundamentals of Database Systems (2015), "For each regular binary 1:N relationship type R, identify the relation S that represent the participating entity type at the N-side of the relationship type.
  \item Include as foreign key in S the primary key of the relation T that represents the other entity type participating in R.
  \item Include any simple attributes of the 1:N relation type as attributes of S." \cite{elmasri}
\end{itemize}

\subsubsection{Subcategory}

Category (\underline{Name}, Description, \textit{Parent\_Category\_Name})

The 1:N relationship Subcategory is mapped by choosing the participating entity type Category to serve in the role of S, because it is a self-relationship. Moreover, we added the foreign key Parent\_Category\_Name to connect the entity type to itself.

\subsubsection{Contains}

Product (\underline{SKU}, Name, Price, Description, Weight, Brand, Width, Height, Length, \textit{Category\_name})

The 1:N relationship Contains is mapped by choosing the participating entity type Product to serve in the role of S, because its participation in the Contains relationship is from the N-side. Moreover, we added the foreign key Category\_name to connect the two participating entities.

\subsubsection{Supply}

Product (\underline{SKU}, Name, Price, Description, Weight, Brand, Width, Height, Length, \textit{Category\_name}, \\
\textit{Supplier\_website})

The 1:N relationship Supply is mapped by choosing the participating entity type Product to serve in the role of S, because its participation in the Supply relationship is from the N-side. Moreover, we added the foreign key Supplier\_website to connect the two participating entities.

\subsubsection{Works In}

Employee (\underline{SSN}, Position, Salary, Hire\_date, Gender, Date\_of\_birth, Email, First\_name, Last\_name,\\
Phone\_number, Country, State, City, Street, Building, Apartment, \textit{Branch\_phone\_number})

The 1:N relationship Works\_in is mapped by choosing the participating entity type Employee to serve in the role of S, because its participation in the Works\_in relationship is from the N-side. Moreover, we added the foreign key Branch\_phone\_number to connect the two participating entities.

\subsubsection{Supervision}

Employee (\underline{SSN}, Position, Salary, Hire\_date, Gender, Date\_of\_birth, Email, First\_name, \\
Last\_name, Phone\_number, Country, State, City, Street, Building, Apartment, \textit{Branch\_phone\_number}, \\
\textit{Supervisor\_SSN})

The 1:N relationship Supervision is mapped by choosing the participating entity type Employee to serve in the role of S, because it is a self-relationship. Moreover, we added the foreign key Supervisor\_SSN to connect the entity type to itself.

\subsubsection{Physical Checkout}

Order (\underline{Order\_id}, Notes, Payment\_method, Total\_amount, Is\_online, \textit{Employee\_SSN})

The 1:N relationship Physical\_checkout is mapped by choosing the participating entity type Order to serve in the role of S, because its participation in the Physical\_checkout relationship is from the N-side. Moreover, we added the foreign key Employee\_SSN to connect the two participating entities.

\subsubsection{Made By}

Order (\underline{Order\_id}, Notes, Payment\_method, Total\_amount, Is\_online, \textit{Employee\_SSN}, \\
\textit{Customer\_phone\_number}, Date)

The 1:N relationship Made\_by is mapped by choosing the participating entity type Order to serve in the role of S, because its participation in the Made\_by relationship is from the N-side. Moreover, we added the foreign key Customer\_phone\_number to connect the two participating entities. Finally, we added the atomic attribute of the relationship Made\_by to the relation Order.

\subsubsection{Works For}

Employee (\underline{SSN}, Position, Salary, Hire\_date, Gender, Date\_of\_birth, Email, First\_name, Last\_name, \\
Phone\_number, Country, State, City, Street, Building, Apartment, \textit{Branch\_phone\_number}, \\
\textit{Supervisor\_SSN}, \textit{Department\_name})

The 1:N relationship Works\_for is mapped by choosing the participating entity type Employee to serve in the role of S, because its participation in the Works\_for relationship is from the N-side. Moreover, we added the foreign key Department\_name to connect the two participating entities. Finally, we added all the atomic attributes of the relationship Works\_for to the relation Employee.

\subsubsection{Dependents Of}

Dependent (Owner\_SSN, Name, Gender, Date\_of\_birth, Relationship, \textit{Employee\_SSN})

The 1:N relationship Dependents\_of is mapped by choosing the participating entity type Dependent to serve in the role of S, because its participation in the Dependents\_of relationship is from the N-side. Moreover, we added the foreign key Employee\_SSN to connect the two participating entities.

\subsubsection{Assigned To}

Support\_ticket (\underline{Ticket\_id}, Description, Subject, Status, Priority, \textit{Employee\_SSN})

The 1:N relationship Assigned\_to is mapped by choosing the participating entity type Support\_ticket to serve in the role of S, because its participation in the Assigned\_to relationship is from the N-side. Moreover, we added the foreign key Employee\_SSN to connect the two participating entities.

\subsubsection{Delivers}

Order (\underline{Order\_id}, Notes, Payment\_method, Total\_amount, Is\_online, \textit{Employee\_SSN}, \\
\textit{Customer\_phone\_number}, Date, \textit{Driver\_license\_number})

The 1:N relationship Delivers is mapped by choosing the participating entity type Order to serve in the role of S, because its participation in the Delivers relationship is from the N-side. Moreover, we added the foreign key Driver\_license\_number to connect the two participating entities.

\subsubsection{Requests}

Support\_ticket (\underline{Ticket\_id}, Description, Subject, Status, Priority, \textit{Employee\_SSN}, \\
\textit{Customer\_phone\_number})

The 1:N relationship Requests is mapped by choosing the participating entity type Support\_ticket to serve in the role of S, because its participation in the Requests relationship is from the N-side. Moreover, we added the foreign key Custom\_phone\_number to connect the two participating entities.

\subsection{Mapping of Multivalued Attributes}

\begin{itemize}
  \item According to Elmasri and Navathe, in Fundamentals of Database Systems (2015), "For each regular binary M:N relationship type R, create a new relation S to represent R.
  \item Include as foreign key attributes in S the primary keys of the relations that represent the participating entity types; combination will form the primary key of S.
  \item Also include any simple attributes of the M:N relationship type (or simple components of composite attributes) as attributes of S." \cite{elmasri}
\end{itemize}

\subsubsection{Wishlist}

Wishlist (\underline{SKU}, \underline{\textit{Customer\_phone\_number}}, Total\_amount)

The M:N relationship Wishlist from the ER diagram is mapped by creating a relation Wishlist in the relational database schema. The primary keys of the Product and Customer relations are included as foreign keys in Wishlist and renamed Product\_SKU and Customer\_phone\_number, respectively. The attribute Total\_amount represents the total amount of the product in the wish list. The primary key of Wishlist is the tuple of foreign keys \{Product\_SKU, Customer\_phone\_number\}.

\subsubsection{Located In}

Located\_in (\underline{\textit{Product\_SKU}}, \underline{\textit{Branch\_phone\_number}}, Quantity, Shelf\_location)

The M:N relationship type Located\_in from the ER diagram is mapped by creating a relation Located\_in in the relational database schema. The primary keys of the Product and Branch relations are included as foreign keys in Located\_in and renamed Product\_SKU and Branch\_phone\_number, respectively. Attributes Quantity and Shelf\_location in Located\_in represent the Quantity and Shelf\_location attributes of the relation type. The primary key of the Located\_in relation is the combination of the foreign key attributes \{Product\_SKU, Branch\_phone\_number\}.

\subsubsection{Reviews}

Reviews (\underline{\textit{Product\_SKU}}, \underline{\textit{Customer\_phone\_number}}, Review\_date, Rating, Comment, Description)

The M:N relationship type Reviews from the ER diagram is mapped by creating a relation Reviews in the relational database schema. The primary keys of the Product and Customer relations are included as foreign keys in Reviews and renamed Product\_SKU and Customer\_phone\_number, respectively. Attributes Review\_date, Rating, Comment, and Description in Reviews represent the corresponding attributes of the relation type. The primary key of the Reviews relation is the combination of the foreign key attributes \{Product\_SKU, Customer\_phone\_number\}. Furthermore, we didn't include the multi-valued attribute Image\_URLs as we'll create its own relation later.

\subsubsection{Purchased}

Purchased (\underline{\textit{Product\_SKU}}, \underline{\textit{Order\_id}}, Quantity, Amount)

The M:N relationship type Purchased from the ER diagram is mapped by creating a relation Purchased in the relational database schema. The primary keys of the Product and Order relations are included as foreign keys in Purchased and renamed Product\_SKU and Order\_id, respectively. Attributes Quantity and Amount in Purchased represent the corresponding attributes of the relation type. The primary key of the Purchased relation is the combination of the foreign key attributes \{Product\_SKU, Order\_id\}.

\subsection{Mapping of Binary M:N Relationship Types}

\begin{itemize}
  \item According to Elmasri and Navathe, in Fundamentals of Database Systems (2015), "For each multivalued attribute A, create a new relation R.
  \item This relation R will include an attribute corresponding to A, plus the primary key attribute K-as a foreign key in R-of the relation that represents the entity type of relationship type that has A as an attribute.
  \item The primary key of R is the combination of A and K. If the multivalued attribute is composite, we include its simple components." \cite{elmasri}
\end{itemize}

\subsubsection{Colors}

Colors (\underline{\textit{Product\_SKU}}, \underline{Product\_color})

The relation Colors is created. The attribute Product\_color represents the multivalued attribute Colors of Product, while Product\_SKU—as foreign key—represents the primary key of the Product relation. The primary key of Color\_s is the combination of \{Product\_SKU, Product\_color\}.

\subsubsection{Image URLs}

Image\_URLs (\underline{\textit{Product\_SKU}}, \textit{Customer\_phone\_number}, \underline{Product\_Image\_URL})

The relation Image\_URLs is created. The attribute Product\_Image\_URL represents the multivalued attribute Image\_URL of Product, while Product\_SKU—as foreign key—represents the primary key of the Product relation. The primary key of Image\_URLs is the combination of \{Product\_SKU, Product\_Image\_URL\}.

\subsubsection{Working Hours}

Working\_hours (\underline{\textit{Branch\_phone\_number}}, \underline{Day}, \underline{Opening\_hour}, \underline{Closing\_hour})

The relation Working\_hours is created. The attributes Day, Opening\_hour, and Closing\_hour represent the composite-multivalued attribute Work\_hours of Branch, while Branch\_phone\_number—as foreign key—represents the primary key of the Branch relation. The primary key of Working\_hours is the combination of \{Branch\_phone\_number, Day, Opening\_hour, Closing\_hour\}.

\subsubsection{Department Location}

Department\_location (\underline{Department\_name}, \underline{Location})

The relation Department\_location is created. The attribute Location represents the multivalued attribute Locations of Department, while Department\_name—as foreign key—represents the primary key of the Department relation. The primary key of Department\_location is the combination of \\
\{Department\_name, Location\}.
